\input{preamble.tex}
\usepackage{hyperref}
\usepackage{booktabs}
\usepackage{comment}
\usepackage{natbib}
\usepackage{bbm}
\usepackage{mathtools}
\usepackage{amsfonts}
\usepackage{appendix}
\usepackage{csquotes}
\usepackage{titlesec}
\usepackage{amssymb}
\usepackage{listings}
\usepackage{subcaption}
\usepackage{subfig}
\usepackage{floatrow}
\usepackage{float}
\lstset{
    frame = single,
    breaklines=true,
    basicstyle=\ttfamily}
\usepackage{tikz, forest}
\usepackage{natbib}

\DeclareMathOperator*{\maximize}{maximize}
\DeclareMathOperator*{\minimize}{minimize}


\begin{document}

\header{Assignment 2}{Zhang Rongqi}{A0276566M}

\section{Implementation}

\begin{itemize}
    \item \texttt{detect\_lines()}: Followed the instructions to call \texttt{cv2.Canny} and \texttt{cv2.HoughLines} to automatically get horizontal and vertical lines. Number of Lines is $O(L)$.
    
    \item \texttt{get\_lines\_from\_line\_pts()}: Simply do the cross product of each pair of points, and then do the normalization. Time complexity: $O(1)$.
    
    \item \texttt{get\_pairwise\_intersections()}: Iterate each pair of lines from the list of lines and do the cross product to get their intersection and save the result in a list. If the scale factor of the intersection equals $0$, meaning they are parallel lines, skip adding it to the list. Number of points is $O(P)=O(L^2)$. Time complexity: $O(L^2)$.
    
    \item \texttt{get\_support\_mtx()}: From the pairwise intersections and lines, compute the support matrix between the intersections and the lines if the distance between the intersection and line is within the distance threshold. Use $d = \frac{|Ax + By + C|}{\sqrt{A^{2} + B^{2}}}$ to calculate the distance between the line $l=(A,B,C)^T$ and the point $p=(x,y,1)^T$. Time complexity: $O(PL)=O(L^3)$.
    
    \item \texttt{get\_vanishing\_pts()}: For each of the $n$ loops, find the point with maximum supporting lines. And remove these lines to avoid them supporting another vanishing point. Time complexity: $O(nPL)$.
    
    \item \texttt{get\_vanishing\_line()}: Simply do the cross product of each pair of points, and then do the normalization. Time complexity: $O(1)$.
    
    \item \texttt{get\_target\_height()}: First, find the horizontal and vertical vanishing points. Then project $t_2$ onto $t_{1}b_{1}$ to get $\Tilde{t}_2$. Then take the norm of vectors on $t_{1}b_{1}$ to get their distance ratio in the real world. Time complexity: $O(1)$.

\end{itemize}

\section{Result}

Intermediate results I get are the same as the TA's. Due to space limitations, the results will not be displayed. Final result: \texttt{INFO: the target height is 3.44m.}

\end{document}

\end{document}
